\chapter{Introduction}
Sharing knowledge is an important ability, that humans and not only humans, developed via evolution.
It is such a powerful skill mainly because you don't need to fight a tiger if you know when it will come and leave before that happens.  
In a way, our whole world is just a compilation of physical senses which is also information. Once it's obvious the whole species working on the same thing -- information.
At first simple signals, screams, signs, drawings. Later language, clothing, symbolic systems. Then post offices, telegraph, phone, Internet.
The whole purpose of humankind started to grow around the idea of knowing to do "what?" and "when?". Huge evolution algorithm that collects data about how to live longer and happier. 
But when you collect data, you need somehow to store it and to share it. Because it may be easy with a small amount of information like a presence of a tiger, but it becomes harder and harder.
Now we have huge complex systems that answer our questions. Sadly computers now can answer our questions much faster than we can ask or understand the answer. Of course, at some point, we will develop a neural interface that will make our IO magically fast.  Or maybe a computer that will ask other computer questions, so we will need to only hear the answers to our abstract questions like: "Why am I such an awesome human and dolphins are stupid?".
Until then we need to work on our data transfer and get the most from our eyes, noses, ears, etc. 
Eyes already have most of our attention \cite{Hutmacher}, so my choice lands on ears and hearing. So how to pass information by hearing, of course, there's language and speech, but it's a more appropriate topic for a linguist and not something new. Our task is to find some semiotic apparatus of sound similar to one for visualization. Data visualization is a set of disciplines that studies ways to communicate data graphically. Same for a sound there is Sonification.