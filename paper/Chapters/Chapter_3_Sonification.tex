\chapter{Sonification}
To give a definition we should decide the terminology. There are three words that describe communicating data via sound without verbal signals. Audification, sonification, and auditory display. But for simplicity, we will work with the term Data Sonification. 
Data sonification has a lot of different definitions, mostly because it's hard to draw a line where sound representation is not a sonification anymore. 
Obviously, most people when to hear sonification and probably visualization will think about some direct mapping of data to some sound parameter, for example, the amount of precipitation is the frequency of the sound wave (pitch). But this mapping might be indirect and guided with some complex algorithms.
Or some previous experience can be associated with data representation. Complex models may appear if you decide not to map data, but create a soundscape to give the user a feeling of the data. 
David Worrall in Sonification Design developed his own definition of Data sonification. 
\begin{displayquote}
    "Data sonification is the acoustic representation of informational data for
relational non-linguistic interpretation by listeners, in order that they might
increase their knowledge of the source from which the data was acquired." \cite{Worrall}
\end{displayquote}
It should be noticed that its definition of data sonification and not sonification in general. Worrall combined definitions from multiple previous works to create a boundary for this topic. It is different form the sonification or the auditory display as those terms may have much wider or complex semantic connections. Also, this definition includes "non-linguistic" to separate sonification from linguistics, it may be compared to a less abstract level of programing if language is a programing language then sounds are simple primitive instructions or even electricity itself. The second important part of the definition is "might increase the knowledge of the source from which the data was acquired" it is important to state that the aim of data sonification is to communicate information gathered from raw data. Of course, this description doesn't explain the ways data can be communicated through sound. Mainly this is because sonification has its own separate types and sound has many parameters to tweak.
I will describe sonification types based on two sonification books \cite{Worrall} and \cite{Hermann}